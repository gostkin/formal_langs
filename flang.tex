\documentclass[11pt,a4paper]{report}
\usepackage[a4paper, left=20mm, right=20mm, top=30mm, bottom=20mm]{geometry}
\usepackage{amsmath,amsfonts,amssymb,amsthm,epsfig,epstopdf,titling,url,array}
\usepackage[T2A]{fontenc}
\usepackage[T1]{fontenc}
\usepackage[utf8]{inputenc}
\usepackage[russian]{babel}
\usepackage{xparse}
\usepackage[shortlabels]{enumitem}


\theoremstyle{definition}
\newtheorem{task}{Задача}[section]

\theoremstyle{definition}
\newtheorem{theorem}{Теорема}[section]
\newtheorem{lemma}{Лемма}[section]
\newtheorem{preposition}{Предложение}[section]
\newtheorem*{corollary}{Следствие}

\theoremstyle{definition}
\newtheorem{definition}{Определение}[section]
\newtheorem{example}{Пример}[section]


\title{\textbf{Теория формальных языков}}
\author{Госткин Евгений Михайлович\\группа 695}
\date{\today}
\begin{document}
\setlength{\parindent}{1cm}
\maketitle
\tableofcontents
\newpage
\chapter{Конечные автоматы}
\section{Введение}
\definition{$\Sigma = \{a_1, \ldots, a_n\}$ - алфавит.}
\definition{$\omega\in\Sigma^*$ - слово.}
\definition{$L\subset\Sigma^*$ - язык (множество слов).}
\definition{$"\cdot"$ - конкатенация, работает для слов и для языков, $ab \cdot ca = abca$, $L_1~\cdot~L_2~=~\{w_1w_2 | w_1\in L_1, w_2 \in L_2\}$.}
\definition{Степень языка: $L^0 = \{\varepsilon\}, L^1 = L, L^k = \underbrace{L\cdot\ldots\cdot L}_{\text{k раз}}$.}
\definition{Замыкание Клини - $L^* = \bigcup_{k=0}^{\infty}L^k$.}
\section{Конечный автомат}
\definition{Конечный автомат - $M:= \langle Q, \Sigma, \Delta, q_0, F\rangle$.}
\begin{enumerate}
\item{$Q$ - конечное множество, $q \in Q$ - состояния}
\item{$\Sigma$ - алфавит, $|\Sigma| < \infty$}
\item{$\Delta$ - переход, $\Delta\subseteq Q\times\Sigma^*\times Q$, $\langle q_1, u\rangle\rightarrow q_2$}
\item{$q_0\in Q$ - стартовое состояние}
\item{$F\subseteq Q$ - завершающие состояния}
\end{enumerate}
\definition{$\langle q, u\rangle : q\in Q, u \in \Sigma^k$ - конфигурация конечного автомата.}
\definition{Отношение перехода $\vdash_M$ - наименьшее транзитивное рефлекисвное отношение, согласованное с операциями $(\langle q_1, u\rangle\rightarrow q)\in \Delta\implies \langle q_1, u\omega\rangle\vdash_M \langle q_2, \omega\rangle$.}
\definition{$L(M) = \{\omega | \exists q \in F : \langle q_0, \omega\rangle \vdash_M \langle q, \varepsilon\rangle\}$ - язык, задаваемый автоматом.}
\definition{Автоматный язык - язык, задаваемый автоматом.}
\preposition{\label{oneletterway}Любой автоматный язык распознается автоматом с неболее чем однобуквенными переходами.}\newline
$\blacktriangle$ Пусть $\exists u \in \Sigma^*, k:= |u|, 1 < k < \infty$ и $\exists q_1, q_2: (\langle q_1, u \rangle\rightarrow q_2)\in \Delta$. Запишем $u$ в виде $u_1\cdot\ldots\cdot u_k$, где $u_i \in \Sigma$ для $i\in\{1, \ldots, k\}$. Введем новые состояния 
$q_1^\prime, \ldots, q_{k-1}^\prime$ и добавим соответствующие переходы в $\Delta$, удалив прямой переход $(\langle q_1, u \rangle\rightarrow q_2)\in \Delta$, так, чтобы:
\[ \langle q_1, u_1\cdot\ldots\cdot u_k\omega \rangle\vdash_M \langle q_1^\prime, u_2\cdot\ldots\cdot u_k\omega\rangle, \ldots,
\langle q_{k-1}^\prime, u_k\omega \rangle\vdash_M \langle q_2, \omega\rangle\]
Таким образом, мы получаем, что переходы в наши промежуточные состояния теперь происходят ровно по одной букве. Также данное доказательство может быть легко получено построением рисунка.~$\blacksquare$
\preposition{В условиях Предложения~\ref{oneletterway} можно считать, что завершающее состоянияе ровно одно, т. е. $|F| = 1$.}
\newline
$\blacktriangle$  Создадим новое состояние $Q$ и $\forall q \in F $ добавим в $\Delta$ переходы $\langle q, \varepsilon \rangle\rightarrow Q$. Очевидно, что переход по $\varepsilon$ из всех старых завершающих состояний дает нам одно новое конечное состояние $Q$.~$\blacksquare$
\theorem{Всякий автоматный язык распознается автоматом с однобуквенными переходами.}\newline
$\blacktriangle$ Пусть $M = \langle Q, \Sigma, \Delta, q_0, F\rangle$ - конечный автомат с однобуквенными переходами, $M^\prime = \langle Q, \Sigma, \Delta^\prime, q_0, F^\prime\rangle$,
$L = L(M)$. Запись $\Delta(q, u)$ обозначает состояние, в которое можно придти из $q$ по $u$.
\begin{enumerate}
\item{$
(\langle q_1, a \in \Sigma\rangle\rightarrow q_2) \in \Delta^\prime \Leftrightarrow\exists q_3 \in \Delta(q_1, \varepsilon) : (\langle q_3, a\rangle \rightarrow q_2) \in \Delta
$}
\item{$
F^\prime = \{q^\prime | \exists q \in F: q\in \Delta(q^\prime, \varepsilon)\}
$}
\item{$
\omega = q_1,\ldots, q_m \in L(M^\prime)\Leftrightarrow \langle q_0;a_1, \ldots, a_m\rangle\vdash_{M^\prime}\langle q^\prime, \varepsilon\rangle$
$
\Leftrightarrow \exists q_1,\ldots, q_{m-1}\in Q~ \exists q^\prime\in F^\prime: $
\[
\begin{pmatrix} 
\langle q_0, a_1\rangle\rightarrow q_1\\
\ldots\\
\langle q_{m-2}, a_{m-1}\rangle\rightarrow q_{m-1}\\
\langle q_{m-1}, a_m\rangle\rightarrow q^\prime
\end{pmatrix}\in \Delta^\prime \Leftrightarrow
\exists q_1, \ldots, q_{m-1} \exists r_1, \ldots r_{m-1}, r_m:
\exists q^\prime \exists q\in F:
\]
\[
\begin{pmatrix}
r_1\in \Delta(q_0, \varepsilon), \langle r_1, a_1\rangle\rightarrow q_1\in\Delta\\
r_2\in \Delta(q_1, \varepsilon), \langle r_2, a_2\rangle\rightarrow q_2\in\Delta\\
\cdots\\
r_m\in \Delta(q_{m-1}, \varepsilon), \langle r_m, a_m\rangle\rightarrow q^\prime\in\Delta\\
\end{pmatrix}
\]}
\item{$q\in\Delta(q^\prime, \varepsilon)\Leftrightarrow \omega=a_1,\ldots, a_m\in L(M)$ $\blacksquare$}
\end{enumerate}
\section{Детерминированный конечный автомат}
\definition{Детерминированный конечный автомат - ДКА: $M = \langle Q, \Sigma, \Delta, q_0, F\rangle$.}
\begin{enumerate}
\item{$\Delta~\subseteq~Q\times~\Sigma~\times~Q$}
\item{$\forall q_1 \in Q \forall a \in \Sigma \left|\{q_2 ~|~ \langle q_1, a\rangle\rightarrow q_2\}\right|\leq 1$}
\end{enumerate}

\theorem{Любой автоматный язык реализуется ДКА.}
\begin{enumerate}
\item{$ L = L(M), M=\langle Q, \Sigma, \Delta, q_0, F\rangle - НКА с однобуквенными переходами, M^\prime = \langle 2^Q, \Sigma, \Delta^\prime, \{q_0\}, F^\prime \rangle$}
\item{$(\langle Q_1, a\rangle\rightarrow Q_2)\in \Delta^\prime \Leftrightarrow Q_2=\{q_2~|~\exists q_1\in Q_1:\langle q_1, a\rangle\rightarrow q_2\in \Delta\}$}
\item{$F = \{Q^\prime~|~Q^\prime\cap F\neq \varnothing\}$}

\end{enumerate}
\preposition{$\Delta^\prime(\{q_0\},\omega)=\{\Delta(q_0,\omega)\}$.}\newline
Индукция по $|\omega|$:
\begin{description}
\item[База:]{$\omega = \varepsilon$, $\Delta(\{q_0\}, \varepsilon) = \{q_0\} = \Delta(q_0, \varepsilon)$}
\item[Шаг:]{$|w|=n \Rightarrow \omega = \omega^\prime a, $ где $a \in \Sigma, |\omega^\prime| < n$}
\[
\Delta^\prime(\{q_0\}, \omega^\prime a) = \Delta ^\prime(\Delta^\prime(\{q_0\}, \omega^\prime), a)= \Delta^\prime(\Delta(q_0, \omega^\prime), a) =\footnote{По определению перехода $\Delta^\prime$} \Delta(\Delta(q, \omega^\prime), a) = \Delta(q_0, \omega^\prime a) 
\]
\[
\omega\in L(M^\prime)\Leftrightarrow\Delta^\prime(\{q_0\}, \omega)\subseteq F^\prime \Leftrightarrow \Delta(q_0, \omega)\in F^\prime\Leftrightarrow \Delta(q_0, \omega)\cap F \neq \varnothing\Leftrightarrow \omega\in L(M)
\]
\end{description}
\corollary{Любой автономный язык можно распознать за $O(n)$ времени и $O(1)$ памяти.}
\theorem{Автоматные языки замкнуты относительно конкатенации, объединения, итерации.}\newline
$\blacktriangle$ Пусть $L_1$ и $L_2$ произвольные языки.
\begin{enumerate}
\item{Создаем переход по $\varepsilon$ из завершающего состояния автомата, задающего $L_1$, в $L_2$.}
\item{Создаем новое начальное состояние с переходами по $\varepsilon$ в начальное состояние автомата, задающего $L_1$, и в начальное состояние автомата, задающего $L_2$.}
\item{Создаем новое начальное состояние с переходом по $\varepsilon$ в начальное состояние автомата, задающего $L_1$, и переход по $\varepsilon$ из конечного состояния автомата, задающего $L_1$, в новое начальное состояние. $\blacksquare$}
\end{enumerate}
\begin{thebibliography}{0}
\bibitem{pentus}
М.Р. Пентус, А.Е. Пентус -- Математическая теория формальных языков
\bibitem{au} 
А.Ахо, Д. Ульман -- Теория синтаксического анализа, перевода и компиляции
\bibitem{aus} 
А.Ахо, Д. Ульман, Сети - Компиляторы: принципы, построение и анализ
\bibitem{itmo} 
Конспекты ИТМО
\end{thebibliography}
\end{document}

alexey.sorokin@list.ru
