\documentclass[11pt,a4paper]{article}
\usepackage[a4paper, left=20mm, right=20mm, top=30mm, bottom=20mm]{geometry}

\usepackage{amsmath,amsfonts,amssymb,amsthm,epsfig,epstopdf,titling,url,array}
\usepackage[T2A,T1]{fontenc}
\usepackage[utf8]{inputenc}
\usepackage[russian]{babel}
\usepackage{xparse}
\usepackage[shortlabels]{enumitem}
\usepackage{tikz}
\usetikzlibrary{automata,positioning}


\theoremstyle{definition}
\newtheorem{task}{Задача}[section]

\theoremstyle{definition}
\newtheorem{theorem}{Теорема}[section]
\newtheorem{lemma}{Лемма}[section]
\newtheorem{preposition}{Предложение}[section]
\newtheorem{subpreposition}{Предложение}[theorem]
\newtheorem*{corollary}{Следствие}

\theoremstyle{definition}
\newtheorem{definition}{Определение}[section]
\newtheorem{example}{Пример}[section]


\title{\textbf{Теория формальных языков}}
\author{Госткин Евгений Михайлович\\группа 695}
\date{\today}
\begin{document}
\setlength{\parindent}{1cm}
\maketitle
\tableofcontents
\newpage
\section{Введение}
\definition{$\Sigma = \{a_1, \ldots, a_n\}$ - алфавит.}
\definition{$\omega\in\Sigma^*$ - слово.}
\definition{$L\subset\Sigma^*$ - язык (множество слов).}
\definition{$"\cdot"$ - конкатенация, работает для слов и для языков, $ab \cdot ca = abca$, $L_1~\cdot~L_2~=~\{w_1w_2 | w_1\in L_1, w_2 \in L_2\}$.}
\definition{Степень языка: $L^0 = \{\varepsilon\}, L^1 = L, L^k = \underbrace{L\cdot\ldots\cdot L}_{\text{k раз}}$.}
\definition{Замыкание Клини - $L^* = \bigcup_{k=0}^{\infty}L^k$.}
\section{Конечные автоматы и регулярные выражения}
\subsection{Конечные автоматы}
\definition{Конечный автомат - $M:= \langle Q, \Sigma, \Delta, q_0, F\rangle$.}
\begin{enumerate}
\item{$Q$ - конечное множество, $q \in Q$ - состояния}
\item{$\Sigma$ - алфавит, $|\Sigma| < \infty$}
\item{$\Delta$ - переход, $\Delta\subseteq Q\times\Sigma^*\times Q$, $\langle q_1, u\rangle\rightarrow q_2$}
\item{$q_0\in Q$ - стартовое состояние}
\item{$F\subseteq Q$ - завершающие состояния}
\end{enumerate}
\definition{$\langle q, u\rangle : q\in Q, u \in \Sigma^k$ - конфигурация конечного автомата.}
\definition{Отношение перехода $\vdash_M$ - наименьшее транзитивное рефлекисвное отношение, согласованное с операциями $(\langle q_1, u\rangle\rightarrow q)\in \Delta\implies \langle q_1, u\omega\rangle\vdash_M \langle q_2, \omega\rangle$.}
\definition{$L(M) = \{\omega | \exists q \in F : \langle q_0, \omega\rangle \vdash_M \langle q, \varepsilon\rangle\}$ - язык, задаваемый автоматом.}
\definition{Автоматный язык - язык, задаваемый автоматом.}
\preposition{\label{oneletterway}Любой автоматный язык распознается автоматом с неболее чем однобуквенными переходами.}
\begin{proof}[Доказательство]
Пусть $\exists u \in \Sigma^*, k:= |u|, 1 < k < \infty$ и $\exists q_1, q_2: (\langle q_1, u \rangle\rightarrow q_2)\in \Delta$. Запишем $u$ в виде $u_1\cdot\ldots\cdot u_k$, где $u_i \in \Sigma$ для $i\in\{1, \ldots, k\}$. Введем новые состояния 
$q_1^\prime, \ldots, q_{k-1}^\prime$ и добавим соответствующие переходы в $\Delta$, удалив прямой переход $(\langle q_1, u \rangle\rightarrow q_2)\in \Delta$, так, чтобы:
\[ \langle q_1, u_1\cdot\ldots\cdot u_k\omega \rangle\vdash_M \langle q_1^\prime, u_2\cdot\ldots\cdot u_k\omega\rangle, \ldots,
\langle q_{k-1}^\prime, u_k\omega \rangle\vdash_M \langle q_2, \omega\rangle\]
Таким образом, мы получаем, что переходы в наши промежуточные состояния теперь происходят ровно по одной букве. Также данное доказательство может быть легко получено построением рисунка.
\end{proof}
\preposition{В условиях Предложения~\ref{oneletterway} можно считать, что завершающее состоянияе ровно одно, т. е. $|F| = 1$.}
\begin{proof}[Доказательство]
Создадим новое состояние $Q$ и $\forall q \in F $ добавим в $\Delta$ переходы $\langle q, \varepsilon \rangle\rightarrow Q$. Очевидно, что переход по $\varepsilon$ из всех старых завершающих состояний дает нам одно новое конечное состояние $Q$.
\end{proof}
\theorem{Всякий автоматный язык распознается автоматом с однобуквенными переходами.}
\begin{proof}[Доказательство]
Пусть $M = \langle Q, \Sigma, \Delta, q_0, F\rangle$ - конечный автомат с однобуквенными переходами, $M^\prime = \langle Q, \Sigma, \Delta^\prime, q_0, F^\prime\rangle$,
$L = L(M)$. Запись $\Delta(q, u)$ обозначает состояние, в которое можно придти из $q$ по $u$.
\begin{enumerate}
\item{$
(\langle q_1, a \in \Sigma\rangle\rightarrow q_2) \in \Delta^\prime \Leftrightarrow\exists q_3 \in \Delta(q_1, \varepsilon) : (\langle q_3, a\rangle \rightarrow q_2) \in \Delta
$}
\item{$
F^\prime = \{q^\prime | \exists q \in F: q\in \Delta(q^\prime, \varepsilon)\}
$}
\item{$
\omega = q_1,\ldots, q_m \in L(M^\prime)\Leftrightarrow \langle q_0;a_1, \ldots, a_m\rangle\vdash_{M^\prime}\langle q^\prime, \varepsilon\rangle$
$
\Leftrightarrow \exists q_1,\ldots, q_{m-1}\in Q~ \exists q^\prime\in F^\prime: $
\[
\begin{pmatrix} 
\langle q_0, a_1\rangle\rightarrow q_1\\
\ldots\\
\langle q_{m-2}, a_{m-1}\rangle\rightarrow q_{m-1}\\
\langle q_{m-1}, a_m\rangle\rightarrow q^\prime
\end{pmatrix}\in \Delta^\prime \Leftrightarrow
\exists q_1, \ldots, q_{m-1} \exists r_1, \ldots r_{m-1}, r_m:
\exists q^\prime \exists q\in F:
\]
\[
\begin{pmatrix}
r_1\in \Delta(q_0, \varepsilon), \langle r_1, a_1\rangle\rightarrow q_1\in\Delta\\
r_2\in \Delta(q_1, \varepsilon), \langle r_2, a_2\rangle\rightarrow q_2\in\Delta\\
\cdots\\
r_m\in \Delta(q_{m-1}, \varepsilon), \langle r_m, a_m\rangle\rightarrow q^\prime\in\Delta\\
\end{pmatrix}
\]}
\item{$q\in\Delta(q^\prime, \varepsilon)\Leftrightarrow \omega=a_1,\ldots, a_m\in L(M)$}
\end{enumerate}
\end{proof}
\subsection{Детерминированные конечные автоматы}
\definition{Детерминированный конечный автомат - ДКА: $M = \langle Q, \Sigma, \Delta, q_0, F\rangle$.}
\begin{enumerate}
\item{$\Delta~\subseteq~Q\times~\Sigma~\times~Q$}
\item{$\forall q_1 \in Q \forall a \in \Sigma \left|\{q_2 ~|~ \langle q_1, a\rangle\rightarrow q_2\}\right|\leq 1$}
\end{enumerate}

\theorem{Любой автоматный язык реализуется ДКА.}
\begin{proof}[Доказательство]
\begin{enumerate}
\item{$ L = L(M), M=\langle Q, \Sigma, \Delta, q_0, F\rangle - НКА с однобуквенными переходами, M^\prime = \langle 2^Q, \Sigma, \Delta^\prime, \{q_0\}, F^\prime \rangle$}
\item{$(\langle Q_1, a\rangle\rightarrow Q_2)\in \Delta^\prime \Leftrightarrow Q_2=\{q_2~|~\exists q_1\in Q_1:\langle q_1, a\rangle\rightarrow q_2\in \Delta\}$}
\item{$F = \{Q^\prime~|~Q^\prime\cap F\neq \varnothing\}$}

\end{enumerate}
\end{proof}
\preposition{$\Delta^\prime(\{q_0\},\omega)=\{\Delta(q_0,\omega)\}$.}
\begin{proof}[Доказательство]
Индукция по $|\omega|$:
\begin{description}
\item[База:]{$\omega = \varepsilon$, $\Delta(\{q_0\}, \varepsilon) = \{q_0\} = \Delta(q_0, \varepsilon)$}
\item[Шаг:]{$|w|=n \Rightarrow \omega = \omega^\prime a, $ где $a \in \Sigma, |\omega^\prime| < n$}
\[
\Delta^\prime(\{q_0\}, \omega^\prime a) = \Delta ^\prime(\Delta^\prime(\{q_0\}, \omega^\prime), a)= \Delta^\prime(\Delta(q_0, \omega^\prime), a) =\footnote{По определению перехода $\Delta^\prime$} \Delta(\Delta(q, \omega^\prime), a) = \Delta(q_0, \omega^\prime a) 
\]
\[
\omega\in L(M^\prime)\Leftrightarrow\Delta^\prime(\{q_0\}, \omega)\subseteq F^\prime \Leftrightarrow \Delta(q_0, \omega)\in F^\prime\Leftrightarrow \Delta(q_0, \omega)\cap F \neq \varnothing\Leftrightarrow \omega\in L(M)
\]
\end{description}
\end{proof}
\corollary{Любой автономный язык можно распознать за $O(n)$ времени и $O(1)$ памяти.}
\theorem{Автоматные языки замкнуты относительно конкатенации, объединения, итерации.}
\begin{proof}[Доказательство]
Пусть $L_1$ и $L_2$ произвольные языки.
\begin{enumerate}
\item{Создаем переход по $\varepsilon$ из завершающего состояния автомата, задающего $L_1$, в $L_2$.}
\item{Создаем новое начальное состояние с переходами по $\varepsilon$ в начальное состояние автомата, задающего $L_1$, и в начальное состояние автомата, задающего $L_2$.}
\item{Создаем новое начальное состояние с переходом по $\varepsilon$ в начальное состояние автомата, задающего $L_1$, и переход по $\varepsilon$ из конечного состояния автомата, задающего $L_1$, в новое начальное состояние.}
\end{enumerate}
\end{proof}
\definition{Полный ДКА - ДКА, в котором $\forall q_1 \in Q~\forall a \in \Sigma~\exists q_2 \in Q~|~\langle q_1, a\rangle\rightarrow q_2\in \Delta$.}
\preposition{$\forall \omega \in \Sigma^*~\exists! q\in Q: (\langle q_0, \omega\rangle \vdash \langle q, \varepsilon\rangle)$}
\theorem{Для любого ДКА существует эквивалентный полный ДКА.}
\begin{proof}[Доказательство]
Очевидно по рисунку.\ \\
\begin{tikzpicture}[shorten >=1pt, node distance=4cm,on grid,auto] 
   \node[state, initial] (q_0){$q_0$}; 
   \node[state] (q_1) [right=of q_0] {$q_1$}; 
   \node[state] (q_2) [right=of q_1] {$q_2$}; 
   \node[state] (q_3) [below=of q_1] {$q_3$}; 
   
    \path[->] 
    (q_0) edge  node {$b$} (q_1)
    (q_1) edge  node  {$a$} (q_2)
    (q_0) edge [loop above] node {$a$} ()
    (q_1) edge [loop above] node {$c$} ()
    (q_2) edge [bend left] node {$b$} (q_0)
    (q_0) edge [bend right] node {$c$} (q_3)
    (q_2) edge [bend left] node {$a, c$} (q_3)
    (q_3) edge [loop below] node {$a, b, c$} ()
    
    (q_1) edge  [bend left] node {$b$} (q_3);
    %(q_4) edge  node  {$\varepsilon$} (q_5);;
          
\end{tikzpicture}
\end{proof}
\theorem{Автоматные языки замкнуты относительно:
\begin{enumerate}[a)]
\item{дополнения,}
\item{пересечения.}
\end{enumerate}
}
\begin{proof}[Доказательство]
$
L = L(M), M = \langle Q, \Sigma, \Delta, q_0, F\rangle - \text{ПДКА}, \overline{M} = \langle Q, \Sigma, \Delta, q_0, Q-F\rangle$.
\begin{enumerate}[a)]
\item{$\omega \in L(\overline{M})\Leftrightarrow\exists q \in (Q - F): (\langle q_0, \omega\rangle\vdash_{\overline{M}}\langle q, \varepsilon\rangle)\Leftrightarrow \exists q\notin F: (\langle q_0, \omega\rangle\vdash{M}\langle q, \varepsilon\rangle)\Leftrightarrow\footnote{Работает только в полном автомате (есть переходы по любому слову).}\omega\notin L(M)$}
\item{Не совсем формальное доказательство: $L_1\cap L_2 = \overline{\overline{L_1}\cup \overline{L_2}}$.\\
Формально:
\begin{enumerate}[1)]
\item{$L_1 = L(M_1), M_1 = \langle Q_1, \Sigma, \Delta_1, q_{01}, F_1\rangle$}
\item{$L_2 = L(M_2), M_2 = \langle Q_2, \Sigma, \Delta_2, q_{02}, F_2\rangle$}
\item{$M = \langle Q_1\times Q_2, \Sigma, \Delta, (q_{01}, q_{02}), F_1\times F_2\rangle$}
\item{$\Delta = \{\langle (q_1, q_2), a\rangle\rightarrow(q_1^\prime, q_2^\prime)~|~((\langle q_1, a\rangle\rightarrow q_1^\prime)\in \Delta_1) \wedge((\langle q_2, a\rangle\rightarrow q_2^\prime)\in \Delta_2)\}$}
\end{enumerate}
\subpreposition{В новом автомате $\langle(q_1, q_2), \omega\rangle\vdash_M\langle(q_1^\prime, q_2^\prime), \varepsilon\rangle\Leftrightarrow\langle q_1,\omega\rangle\vdash_{M_1}\langle q_1^\prime, \varepsilon\rangle\wedge\langle q_2,\omega\rangle\vdash_{M_2}\langle q_2^\prime,\varepsilon\rangle.$}
\begin{proof}[Доказательство]
Индукция по $|\omega|$.
\end{proof}
}
$\omega \in L(M)\Leftrightarrow\exists( q_1, q_2)\in F: (\langle(q_{01}, q_{02}),\omega\rangle\vdash\langle(q_1, q_2),\varepsilon\rangle)\Leftrightarrow\exists q_1\in F_1, \exists q_2\in F_2: (\langle q_{01}, \omega\rangle\vdash_{M_1}\langle q_1, \varepsilon\rangle, \langle q_{02}, \omega\rangle\vdash_{M_2}\langle q_2,\varepsilon\rangle)\Leftrightarrow\omega\in L(M_1)\wedge\omega\in L(M_2) $.
\end{enumerate}
\end{proof}
\subsection{Регулярные выражения}
\definition{$Reg(\Sigma)$ - регулярное выражение над алфавитом $\Sigma$ - наименьшее множество такое, что:
\begin{enumerate}[1)]
\item{$\forall a \in \Sigma: (a\in Reg(\Sigma)) $}
\item{$0, 1 \in Reg(\Sigma)$}
\item{$\forall \alpha, \beta \in Reg(\Sigma): (\alpha \cdot \beta)\in Reg(\Sigma) \wedge (\alpha + \beta) \in Reg(\Sigma)$}
\item{$\forall \alpha \in Reg(\Sigma): (\alpha^* \in Reg(\Sigma))$}
\end{enumerate}
$L(\alpha)\subseteq \Sigma^*$ - язык, задаваемый $\alpha$.}
\theorem{(Клини) Классы регулярных и автоматных языков совпадают.}
\begin{proof}
\begin{description}
\item[Reg $\subseteq$ Aut:]{
Индукция по построению регулярного выражения:
\begin{description}
\item[База:]
{\ \\
\begin{tikzpicture}[shorten >=1pt, node distance=2cm,on grid,auto] 
   \node[state, initial, initial text=a:] (q_0){$q_0$}; 
   \node[state] (q_1) [right=of q_0] {$q_1$}; 
   \node[state,accepting](q_1) [right=of q_0] {$q_1$};
    \path[->] 
    (q_0) edge  node {$a$} (q_1);
\end{tikzpicture}
\ \\
\begin{tikzpicture}[shorten >=1pt, node distance=2cm,on grid,auto] 
   \node[state, initial, initial text=$1$:] (q_0){$q_0$}; 
   \node[state,accepting](q_0) [] {$q_0$};
    \path[->] 
    ;
\end{tikzpicture}
\ \\
\begin{tikzpicture}[shorten >=1pt, node distance=2cm,on grid,auto] 
   \node[state, initial, initial text=$0$:] (q_0){$q_0$}; 
    \path[->] 
    ;
\end{tikzpicture}
}
\item[Шаг:]{Следует из свойств замкнутости из прошлой лекции. }
\end{description}
}
\item[Aut $\subseteq$ Reg:]{
\definition{Регулярный автомат - автомат, метки переходов которого - регулярные выражения.}

Регулярный автомат принимает слово, если оно подходит под конкатенацию регулярных выражений на каком-либо пути от начального состояния до конечного.
\subpreposition{Всякий язык, распознаваемый регулярным автоматом, регулярен.}
\begin{proof}[Доказательство]
Можно считать, что:
\begin{enumerate}
\item{В автомате ровно одно завершающее состояние}
\item{Между любыми двумя состояниями не более одного ребра.\ \\
\begin{center}
\begin{tabular}{c | c}
{\begin{tikzpicture}[shorten >=1pt, node distance=2cm,on grid,auto] 
   \node[state, initial] (q_0) {$q_0$};
   \node[state] (q_1) [right=of q_0] {$q_1$}; 
    \path[->] 
    (q_0) edge [bend left] node {$\alpha$} (q_1)
     (q_0) edge [bend right] node {$\beta$} (q_1);
\end{tikzpicture}} & 
{\begin{tikzpicture}[shorten >=1pt, node distance=3cm,on grid,auto] 
   \node[state, initial] (q_0) {$q_0$};
   \node[state] (q_1) [right=of q_0] {$q_1$}; 
    \path[->] 
    (q_0) edge node {$\alpha + \beta$} (q_1);
\end{tikzpicture}}
\end{tabular}
\end{center}}
\end{enumerate}
Индукция по числу состояний:
\begin{description}
\item[База:]{
\item{\begin{tabular}{c | c}{\begin{tikzpicture}[shorten >=1pt, node distance=2cm,on grid,auto] 
   \node[state, initial,initial text=$\text{1 состояние:}$, accepting] (q_0) {$q_0$};
    \path[->]
     (q_0) edge [loop above] node {$\alpha$} ();
\end{tikzpicture}}
& \LARGE{$\alpha^*$} 
\end{tabular}}
\item{\begin{tabular}{c | c}{\begin{tikzpicture}[shorten >=1pt, node distance=2cm,on grid,auto] 
   \node[state, initial,initial text=$\text{2 состояния:}$] (q_0) {$q_0$};
   \node[state, accepting] (q_1) [right=of q_0] {$q_1$};
    \path[->]
     (q_0) edge node {$\alpha$} (q_1)
     (q_1) edge [bend left] node {$\beta$} (q_0)
     (q_0) edge [loop above] node {$\gamma$} ()
     (q_1) edge [loop above] node {$\delta$} ();
\end{tikzpicture}} & \LARGE{$(\gamma + \alpha\delta^*\beta)^*\alpha\delta^*$}\end{tabular}}
}
\item[Шаг:]{Уменьшим число состояний, удалив произвольное состояние, не являющееся начальным или конечным.
\begin{description}
\item[Было:]{
\begin{tikzpicture}[shorten >=1pt, node distance=2cm,on grid,auto] 
   \node[state, initial,initial text=] (q_0) {$q_0$};
      \node[state] (q_1) [right=of q_0] {$q_1$};
   \node[state, accepting] (q_2) [right=of q_1] {$q_2$};
    \path[->]
    (q_1) edge [loop below] node {$\beta$} ()
    (q_1) edge node {$\gamma$} (q_2)
    (q_0) edge node {$\alpha$} (q_1);

\end{tikzpicture}
}
\item[Получим:]{
\begin{tikzpicture}[shorten >=1pt, node distance=3cm,on grid,auto] 
   \node[state, initial,initial text=] (q_0) {$q_0$};
   \node[state, accepting] (q_2) [right=of q_0] {$q_2$};
    \path[->]
    (q_0) edge node {$\alpha\beta^*\gamma$} (q_2);

\end{tikzpicture}
}
\end{description}
}
\end{description}
Так как $q_1$ - промежуточное, то принимаемый язык не меняется.
\end{proof}
}
\end{description}
\end{proof}
\definition{$L\subseteq \Sigma^*$, тогда $u \sim_L v\Leftrightarrow\forall \omega \in \Sigma^*:(u\omega\in L \Leftrightarrow v\omega\in L)$}
\preposition{Пусть $M$ - ДКА, тогда $\Delta(q_0, u) = \Delta(q_0, v) \implies u\sim_L v$}
\begin{proof}
\[
\forall \omega:\Delta(q_0, u\omega) = \Delta(q_0, v\omega)\implies(u\omega\in L\Leftrightarrow v\omega\in L)
\]
\end{proof}
\theorem{$L$ - автомат $\Leftrightarrow$ $\left|\Sigma^*_{/{\sim_L}}\right|<\infty$}
\begin{proof}
\begin{description}
\item[$\Rightarrow$]{$L = L(M), M$ - ПДКА.
$
\Delta(q_0, u) = \Delta(q_0, v) \Rightarrow u\sim_L v \Rightarrow \left|\Sigma^*_{\sim_L}\right| \leq \left|Q\right|<\infty
$}
\item[$\Leftarrow$]{Рассмотрим автомат на классах эквивалентности:
\begin{enumerate}[1)]
\item{$M = \langle Q, \Sigma, \Delta, q_0, F\rangle$}
\item{$Q = \Sigma^*_{\sim_L}$}
\item{$q_0 = [\varepsilon]$}
\item{$F = \{[\omega]~|~\omega\in L\}$}
\item{$\Delta = \{\langle [\omega], a\rangle\rightarrow[\omega a]~|~\omega\in\Sigma^*, a\in \Sigma\}$}
\end{enumerate}
\begin{description}
\item[Корректность:]{$\omega_1\sim\omega_2 ([\omega_1]=[\omega_2])\Rightarrow \forall a\in\Sigma [\omega_1 a]=[\omega_2 a]$
\[
\omega_1\sim\omega_2\Leftrightarrow\forall u\in\Sigma^* (\omega_1 i\in L \Leftrightarrow\omega_2 u\in L)\Leftrightarrow \forall u^\prime\in\Sigma^* (\omega_1 a  u^\prime\in L \Leftrightarrow \omega_2 a u^\prime\in L)\Leftrightarrow \omega_1 a \sim\omega_2 a
\]}
\end{description}
\subpreposition{$\Delta([u], v) = [uv]$}
\begin{proof}
Индукция по длине $v$:
\begin{description}
\item[База:]{
\begin{enumerate}
\item{$v=\varepsilon\Rightarrow \Delta([a],\varepsilon) = [u]=[u\varepsilon]$}
\item{$v = a\Rightarrow\Delta([u], a) = [ua]$}
\end{enumerate}}
\item[Шаг:]{
\begin{enumerate}[1)]
\item{$v = v^\prime u$}
\item{$\Delta([u], v^\prime a) = \Delta(\Delta([u],v^\prime), a) = \Delta([uv^\prime], a)=[uv^\prime a]=[uv]$}
\end{enumerate}}
\end{description}
\end{proof}}
\[\omega\in L \Leftrightarrow\Delta(q_0 (=[\varepsilon]),\omega)\in F \Leftrightarrow[\omega]\in F \Leftrightarrow\exists u\in L: (\omega\sim u) \Leftrightarrow\omega\in L\]
Заметим, что если $u\in L, \omega \notin L\Rightarrow u\sim \omega.$

\end{description}
\end{proof}
\definition{$M$ - минимальный ПДКА для $L$, если $\nexists M^\prime$ - ПДКА: $L(M^\prime)=L$ такой, что в $M^\prime$ меньшее число состояний.}
\definition{$M$ - ДКА, $M=\langle Q,\Sigma, \Delta,q_0, F\rangle, q_1\sim q_2\Leftrightarrow\forall\omega\in\Sigma^*(\Delta(q_1,\omega)\in F\Leftrightarrow\Delta(q_2,\omega)\in~F)$}
\preposition{Пусть $M = \langle Q,\Sigma, \Delta, q_0, F\rangle, \omega_1,\omega_2\in\Sigma^* \Delta(q_0,\omega_1)\sim\Delta(q_0,\omega_2)\Rightarrow\omega_1\sim_L\omega_2$.}
\begin{proof}
$\Delta(q_0,\omega_1)=q_1\sim q_2=\Delta(q_0, \omega_2): \forall u :\Delta(q_1,u)\in F\Leftrightarrow\Delta (q_2, u)\in F.$\\
\[(\Delta(q_1, u)=\Delta(q_0, \omega_1 u) \in F\wedge \Delta(q_2, u)=\Delta(q_0,\omega_2 u)\in F)\Leftrightarrow(\omega_1 u\in L\Leftrightarrow\omega_2 u\in L)\Rightarrow \omega_1\sim\omega_2.\]
\end{proof}
\newpage
\begin{thebibliography}{0}
\addcontentsline{toc}{section}{\bibname}
\bibitem{pentus}
М.Р. Пентус, А.Е. Пентус -- Математическая теория формальных языков
\bibitem{au} 
А.Ахо, Д. Ульман -- Теория синтаксического анализа, перевода и компиляции
\bibitem{aus} 
А.Ахо, Д. Ульман, Сети - Компиляторы: принципы, построение и анализ
\bibitem{itmo} 
Конспекты ИТМО
\end{thebibliography}
\end{document}

alexey.sorokin@list.ru
